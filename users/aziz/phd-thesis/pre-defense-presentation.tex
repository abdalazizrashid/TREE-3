% Created 2024-05-28 Tue 15:03
% Intended LaTeX compiler: pdflatex
\documentclass[presentation]{beamer}
\usepackage[utf8]{inputenc}
\usepackage[T1]{fontenc}
\usepackage{graphicx}
\usepackage{longtable}
\usepackage{wrapfig}
\usepackage{rotating}
\usepackage[normalem]{ulem}
\usepackage{amsmath}
\usepackage{amssymb}
\usepackage{capt-of}
\usepackage{hyperref}
\usetheme{Madrid}
\author{Al-Maeeni Abdalaziz}
\date{\today}
\title{Presentation}
\subtitle{Pre-defense}
\hypersetup{
 pdfauthor={Al-Maeeni Abdalaziz},
 pdftitle={Presentation},
 pdfkeywords={},
 pdfsubject={},
 pdfcreator={Emacs 30.0.50 (Org mode 9.6.15)}, 
 pdflang={English}}
\begin{document}

\maketitle
\begin{frame}{Outline}
\tableofcontents
\end{frame}


\section{This is the first structural section}
\label{sec:orge74266a}

\begin{frame}[label={sec:orga7c1f3e}]{Frame 1}
\begin{columns}
\begin{column}{0.48\columnwidth}
\begin{block}{Thanks to Eric Fraga}
for the first viable Beamer setup in Org
\end{block}
\end{column}
\begin{column}{0.48\columnwidth}
\begin{block}<2->{Thanks to everyone else}
for contributing to the discussion
\note{This will be formatted as a beamer note
}
\end{block}
\end{column}
\end{columns}
\end{frame}



\section{This is the first structural sectionmeeeeeeee}
\label{sec:org6a770f4}

\begin{frame}[label={sec:orgd89b75b}]{Frame 1}
\begin{columns}
\begin{column}{0.48\columnwidth}
\begin{block}{Thanks to Eric Fraga}
for the first viable Beamer setup in Org
\end{block}
\end{column}
\begin{column}{0.48\columnwidth}
\begin{block}<2->{Thanks to everyone else}
for contributing to the discussion
\note{This will be formatted as a beamer note
}
\end{block}
\end{column}
\end{columns}
\end{frame}


\begin{frame}[label={sec:orgb73677f}]{A more complex slide}
This slide illustrates the use of Beamer blocks.  The following text,
with its own headline, is displayed in a block:
\begin{theorem}[Org mode increases productivity]
\begin{itemize}
\item org mode means not having to remember \LaTeX{} commands.
\item it is based on ascii text which is inherently portable.
\item Emacs!
\end{itemize}

\hfill \(\qed\)
\end{theorem}
\end{frame}
\end{document}
